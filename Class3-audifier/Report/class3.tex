% requirements:
%	TeX environments: TeXlive/MacTeX or MiKTeX
%	compiler: XeLaTeX
%
% ---------------------------------------------------------------------------- %
%                                   preamble                                   %
% ---------------------------------------------------------------------------- %
\documentclass[a4paper]{article}
% 10.5pt = 5 hao

% ---------------------------------- layout ---------------------------------- %
\usepackage[a4paper,top=2.5cm,bottom=2.5cm,left=3cm,right=3cm,% margins
			headheight=1.5cm,headsep=1.5em,
			footskip=2em,
			]{geometry}

% ------------------------------- math symbols ------------------------------- %
% 载入常用的数学包, 符号包
\usepackage{amsmath}
\usepackage{amsfonts}
\usepackage{amssymb}
\usepackage{mathrsfs}
\usepackage{blindtext}

%----------------------------------------------------------------%
%% linespace 行间距,段间距等等
\usepackage{setspace}
% \usepackage{indentfirst} % then the first line of each title should start with a indent.
% 定义标题和段落样式
% 定义1.5倍行距
\renewcommand{\baselinestretch}{1.62}
\setlength{\baselineskip}{12pt}   % set the fixed value of the lineskip
\setlength\parskip{\baselineskip} % set the space between the paragraphs, set the variable \parskip \baselineskip
% parindent
\setlength{\parindent}{0pt}

%----------------------------------------------------------------%
% fonts (style, color, size). 字体的大小,颜色,以及定义常用的字号
\usepackage{ctex}		 	% If you are lazy, the CTEX suit is enough.
% Chinese font
\usepackage{xeCJK}		 	% For the Chinese through XeLaTex
\setCJKmainfont{SimSun} 	% set the mainfont of Chinese as songti. (serif) for
\setCJKsansfont{SimSun} 	% sans serif font for \textsf
\setCJKmonofont{SimSun} 	% monospace font for \texttt
% \punctstyle{kaiming}  	% Remove the space used by symbols like comma. Special for the mainland students like us HUSTers.
\setCJKfamilyfont{song}{SimSun}                       % 宋体 song
\newcommand{\song}{\CJKfamily{song}}
\setCJKfamilyfont{kai}{KaiTi}                         % 楷体  kai
\newcommand{\kai}{\CJKfamily{kai}}
\setCJKfamilyfont{hwzs}{STZhongsong}                  % 华文中宋  hwzs
\newcommand{\hwzs}{\CJKfamily{hwzs}}
\setCJKfamilyfont{hei}{SimHei}                        % 黑体  hei
\newcommand{\hei}{\CJKfamily{hei}}
% English font
\usepackage{fontspec}% Then you can use the fonts installed at your device.
\setmainfont{Times New Roman}
\setsansfont{Times New Roman}
\setmonofont{Times New Roman}
%\setsansfont{[foo.ttf]} % for the fonts at this default path.
% font Color 利用definecolor自己可以定义颜色
\usepackage{xcolor}
\definecolor{MSBlue}{rgb}{.204,.353,.541}
\definecolor{MSLightBlue}{rgb}{.31,.506,.741}
% font Size (I use pinyin represents the corresponding size in Microsorft Word)
% \newcommand{\chuhao}{\fontsize{42pt}{\baselineskip}\selectfont}
\newcommand{\xiaochuhao}{\fontsize{36pt}{\baselineskip}\selectfont}
% \newcommand{\yihao}{\fontsize{28pt}{\baselineskip}\selectfont}
\newcommand{\erhao}{\fontsize{21pt}{\baselineskip}\selectfont}
\newcommand{\xiaoerhao}{\fontsize{18pt}{\baselineskip}\selectfont}
\newcommand{\sanhao}{\fontsize{15.75pt}{\baselineskip}\selectfont}
\newcommand{\sihao}{\fontsize{14pt}{18pt}\selectfont}
\newcommand{\xiaosihao}{\fontsize{12pt}{18pt}\selectfont}
\newcommand{\wuhao}{\fontsize{10.5pt}{18pt}\selectfont}
% \newcommand{\xiaowuhao}{\fontsize{9pt}{\baselineskip}\selectfont}
% \newcommand{\liuhao}{\fontsize{7.875pt}{\baselineskip}\selectfont}
% \newcommand{\qihao}{\fontsize{5.25pt}{\baselineskip}\selectfont}


% ---------------------------------------------------------------------------- %
%% header and footer 页眉,页脚
\usepackage{fancyhdr} % for header and footer

% 设置页眉样式
\newcommand{\headstyle}{
 \fancyhead[C]{ \hwzs\wuhao 华\hspace{0.5em}中\hspace{0.5em}科\hspace{0.5em}技\hspace{0.5em}大\hspace{0.5em}学\hspace{0.5em}课\hspace{0.5em}程\hspace{0.5em}设\hspace{0.5em}计\hspace{0.5em}(报\hspace{0.5em}告)}
}

% 设置页脚样式
\newcommand{\footstyle}{\fancyfoot[C]{\wuhao\thepage}
 \fancyfoot[L]{\rule[5pt]{6.7cm}{0.4pt}}
 \fancyfoot[R]{\rule[5pt]{6.7cm}{0.4pt}}
}
\pagestyle{fancy}
\fancyhf{} % 清空原有样式
\headstyle
\footstyle
% 定义一种新的格式叫做main
\fancypagestyle{main}{%
 \fancyhf{} % 清空原有样式
 \headstyle
 \footstyle
}
\renewcommand{\headrulewidth}{0.4pt}
% \renewcommand{\footrulewidth}{0.4pt}
% \renewcommand{\headrule}{\rule{\textwidth}{0.4pt}}

% ---------------------------------------------------------------------------- %
% set the styles of sections at all levels
% 设置各个标题样式
% 不需要使用part和chapter层级
\usepackage{titlesec}
\usepackage{titletoc}
\titleformat{\section}{\centering\hei\bfseries\xiaoerhao}{\thesection}{1em}{} % 在section标题编号后面加个点
% \titlelabel{\thetitle.\quad} % add a dot after the counter for all levels of sections
% \titleformat*{\section}{\wuhao\bfseries} % 设置标签的形式,5号加粗
\titleformat*{\subsection}{\raggedright\hei\bfseries\sihao}
\titleformat*{\subsubsection}{\raggedright\hei\bfseries\xiaosihao}
\titleformat{\paragraph}[hang]{\raggedright\hei\bfseries\xiaosihao}{\theparagraph}{1em}{}[]

% manual
% \titleformat{command}[shape]{format}{label}{sep}{before-code}[after-code]
% \titlespacing{command}{left}{before-sep}{after-sep}
% 设置新的层级subsubsubsection
\setcounter{tocdepth}{4}
\setcounter{secnumdepth}{4}

% 根据学校要求设置新的section, subsection, subsection,  paragraph

% set the content of section and so on
\newcommand\seccontent{
	\song
	\xiaosihao % 默认五号字体, 行间距为1.5*\baselineskip
    \setlength{\parindent}{2em} % 首段缩进两个M字符
    \setlength{\parskip}{0pt}
    }
\newcommand\tabcontent{
	\song
	\wuhao % 默认五号字体, 行间距为1.5*\baselineskip
	\setlength{\parindent}{2em} % 首段缩进两个M字符
	\setlength{\parskip}{0pt}
}


% ---------------------------------------------------------------------------- %
% for the style of theorems, definitions, proofs and remarks 定义数学里面一些常用的环境
\usepackage{amsthm}
\newtheorem{thm}{\textbf{定理}}[section]
% The section in [] can be replaced by chapter or subsection
\theoremstyle{definition} 
\theoremstyle{plain}
\theoremstyle{remark}

% ---------------------------------------------------------------------------- %
% for the caption and reference 图表及公式的编号规范
\usepackage{float} 		 		  	% table figure positioning
\usepackage{caption}
\captionsetup[figure]{labelformat=default, labelsep=quad,name={图}}
\captionsetup[table]{labelformat=default,labelsep=quad,name={表}}
% 设置图表标题的计数方式
\renewcommand{\thefigure}{\ifnum \thesection>0 \thesection-\fi \arabic{figure}} % set caption label style to 2-1
\renewcommand{\thetable}{\ifnum \thesection>0 \thesection-\fi \arabic{table}} % set caption label style to 2-1
\DeclareCaptionFont{mylabelfont}{\hei\xiaosihao}
\captionsetup[figure]{font=mylabelfont}
\captionsetup[table]{font=mylabelfont}

% 设置图表的autoref的格式
\newcommand{\reffig}[1]{图 \ref{#1}}
\newcommand{\reftab}[1]{表 \ref{#1}}
% 公式的编号格式
\renewcommand\theequation{\arabic{section}-\arabic{equation}}

\usepackage{graphicx} % To include graphixs 添加图所需的包

\graphicspath{{./graphics/}} %设置图片路径
\usepackage{booktabs} % To create three line table including the commands toprule, bottomrule, and midrule
% \usepackage{colortbl} %
% 使用tabularx库并定义新的左右中格式
\usepackage{tabularx}
\usepackage{makecell}
\newcolumntype{L}{X}
\newcolumntype{C}{>{\centering \arraybackslash}X}
\newcolumntype{R}{>{\raggedright \arraybackslash}X}

% ---------------------------------------------------------------------------- %
% set the style of counters 设置计数器
% 设置重新计数的位置
\makeatletter
\@addtoreset{footnote}{page}
\@addtoreset{figure}{section}
\@addtoreset{table}{section}
\@addtoreset{equation}{section}
\makeatother

% ---------------------------------------------------------------------------- %
% tableofcontents, listoftables and listoffigures 目录
%\renewcommand\listfigurename{插图列表}
%\renewcommand\listtablename{表格列表}
%\titlecontents{标题名}[左间距]{标题格式}{标题标志}{无序号标题}{指引线与页码}[下间距]
%\dottedcontents{section}[2.55em]{\song \xiaosihao \bfseries}{2.5em}{1em}
\usepackage{tocloft}
\renewcommand{\contentsname}{\centerline{ \hei\bfseries\xiaoerhao 目\hspace{2em}录}}
\titlecontents{section}[3em]{\song\xiaosihao\bfseries}{\contentslabel{3em}}{\hspace*{-3em}}{\normalfont\titlerule*[8pt]{.}\contentspage}
\titlecontents{subsection}[3em]{\song\xiaosihao}{\contentslabel{3em}}{\hspace*{-3em}}{\titlerule*[8pt]{.}\contentspage}
\titlecontents{subsubsection}[4em]{\song\xiaosihao}{\contentslabel{4em}}{\hspace*{-4em}}{\titlerule*[8pt]{.}\contentspage}
\titlecontents{paragraph}[5em]{\song\xiaosihao}{\contentslabel{5em}}{\hspace*{-5em}}{\titlerule*[8pt]{.}\contentspage}


% ---------------------------------------------------------------------------- %
% 设置声明页
% 使用特殊符号
\usepackage{amssymb}
\usepackage{wasysym}


% ---------------------------------------------------------------------------- %
%	---	定义列表项,列举的样式
\usepackage{enumitem}
\setlist{noitemsep}

% ---------------------------------------------------------------------------- %
% \usepackage{makeindex} For the index 索引
\usepackage{listings} % For the code. 代码
\usepackage{multirow}
% ---------------------------------------------------------------------------- %
% 设置脚注

% ---------------------------------------------------------------------------- %
%% For the hyperlink and bookmark 超链接及书签,(这样生成的pdf中的引用直接点击链接即可到达目的地)
\usepackage[bookmarks=true,colorlinks,linkcolor=black,citecolor=black,urlcolor=purple]{hyperref}% 设置超链接并修改风格


% ---------------------------------------------------------------------------- %
%% For the appendix, 附录
% 设置附录
\usepackage{appendix}
\renewcommand{\appendixname}{附录}

% ---------------------------------------------------------------------------- %
% for the titlepage 标题页,此处可以省略,建议直接使用官方给出的标题页即可
\usepackage{titling}
% 重置命令 maketitle
\renewcommand{\maketitle}{
	\def\HUSTtitlelength{12em}
 	\begin{titlepage}
		\begin{center}
			\vspace*{0em}
			\includegraphics[height=1.61cm]{HUSTlogo.eps}\\
%
			\vspace*{4em}
%
			{\xiaochuhao \hwzs \bfseries 电子线路实验报告}\\
%
			\vspace*{6em}
			{\erhao \hei \bfseries \thetitle}

			\vspace*{6em}
			{\sanhao \hwzs
				\renewcommand\arraystretch{2.7}
				\begin{tabular}{lc}
					\makebox[4em][s]{院 \hfill 系} &
					\underline{\makebox[\HUSTtitlelength]{\school}} \\
					\makebox[4em][s]{专业班级} &
					\underline{\makebox[\HUSTtitlelength]{\classnum}} \\
					\makebox[4em][s]{姓 \hfill 名} &
					\underline{\makebox[\HUSTtitlelength]{\theauthor}} \\
					\makebox[4em][s]{学 \hfill 号} &
					\underline{\makebox[\HUSTtitlelength]{\stunum}} \\
					\makebox[4em][s]{指导教师} &
					\underline{\makebox[\HUSTtitlelength]{\instructor}} \\
			  \end{tabular}
		    }

			\vspace{4em}
			{\sanhao \hwzs \thedate}

		\end{center}
	\end{titlepage}
}


% ------------------------------------ 标题页 ----------------------------------- %
\title{音响放大器的设计} % 论文题目
\def\school{电子信息与通信学院} % 院系
\def\classnum{信卓2201班} % 专业班级
\author{董浩} % 姓名
\def\stunum{U202213781}	% 学号
\def\instructor{陈林} % 指导老师
\date{\today} % 日期


% ---------------------------------------------------------------------------- %
%                                   document                                   %
% ---------------------------------------------------------------------------- %
\begin{document}

\maketitle


\tableofcontents
\thispagestyle{main}

\clearpage
\setcounter{page}{1}
\renewcommand{\thepage}{\arabic{page}}

% ----------------------------------- 主体内容 ----------------------------------- %
\seccontent

\section{实验名称}
音响放大器的设计
\section{实验目的}
\begin{enumerate}
	\item 音响放大器的基本组成
	\item 音调特性控制方法与实现原理
	\item 了解集成功率放大器内部电路工作原理,掌握其外围电路的设计与主要性能参数的测试方法;
	\item 掌握音响放大器的设计方法与电子线路系统的装调技术---综合运用所学知识,进行小型多级电子线路系统的设计与装调。
\end{enumerate}


\section{实验元器件}
\begin{table}[H]
	\centering
	\begin{tabularx}{0.8\textwidth}{|X|X|X|}
		\hline
		名称                    & 型号(参数) & 数量 \\ \hline
		\multirow{2}{*}{集成功放} & LM386  & 1  \\ \cline{2-3}
		                      & NE5532 & 3  \\ \hline
		\multirow{6}{*}{电阻}   & 10KΩ   & 5  \\ \cline{2-3}
		                      & 13KΩ   & 1  \\ \cline{2-3}
		                      & 30KΩ   & 2  \\ \cline{2-3}
		                      & 47KΩ   & 3  \\ \cline{2-3}
		                      & 75KΩ   & 1  \\ \cline{2-3}
		                      & 10Ω 2W & 1  \\ \hline
		\multirow{7}{*}{电容}   & 0.01μF & 2  \\ \cline{2-3}
		                      & 0.22μF & 1  \\ \cline{2-3}
		                      & 0.1μF  & 1  \\ \cline{2-3}
		                      & 1μF    & 1  \\ \cline{2-3}
		                      & 10μF   & 8  \\ \cline{2-3}
		                      & 220μF  & 2  \\ \cline{2-3}
		                      & 470μF  & 1  \\ \hline
		\multirow{2}{*}{电位器}  & 10KΩ   & 3  \\ \cline{2-3}
		                      & 470KΩ  & 2  \\ \hline
		话筒                    & 输出5mV  & 1  \\ \hline
		音乐播放器                 & /      & 1  \\ \hline
	\end{tabularx}
\end{table}
\section{实验任务}
设计一个音响
\subsection{功能要求}
具有话音放大、音调控制、音量控制、卡拉OK伴唱等功能(不含电子混响)。
\subsection{已知条件}
\begin{enumerate}
	\item	集成功放LM386。
	\item	话筒600Ω,输出信号5mV。
	\item	集成运放NE5532。
	\item	10Ω/2W负载电阻1只。
	\item	8Ω/4W扬声器1只。
	\item	音源(MP3 or PC)。
	\item	电源电压±9V(双电源)。
\end{enumerate}

\subsection{技术指标要求}
\begin{enumerate}
	\item	额定功率:$P_o$≥0.3W(γ<3%)
	\item	负载阻抗:$R_L$=10Ω(2W)
	\item	频率响应:$f_L$=50Hz,fH=20kHz
	\item	输入阻抗:$R_i$>>20kΩ
	\item	音调控制特性:1kHz处增益为0dB、125Hz和8kHz处有12dB的调节范围,$\mathrm{A_{VL}=A_{VH}}$≥20dB(选做)
\end{enumerate}

\subsection{测量内容}
\begin{enumerate}
	\item   测量音调控制特性,填入实验表格中,并绘制音调控制特性曲线
	\item	测量频率为1kHz时的输出功率$P_o$及整机电压增益$A_v$,绘制1kHz时的整机输入输出波形
	\item	输入阻抗$R_i$
	\item	输出效率 $\eta$
\end{enumerate}

\section{实验原理}
\subsection{实验电路}
\begin{figure}[H]
	\centering
	\includegraphics[width=1\textwidth]{实验电路}
	\caption{实验电路}
	\label{实验电路}
\end{figure}
\subsection{电路安装与调试技术}
\subsubsection{合理布局,分级装调}
\begin{enumerate}
	\item 音响放大器是一个小型电路系统,安装前要对整机线路进行合理布局。
	\item	一般按照电路的顺序一级一级地布线。
	\item	功放级应远离输入级。
	\item	每一级的地线尽量接在一起。
	\item	连线尽可能短,否则很容易产生自激。
	\item	安装前应检查元器件的质量。
	\item	安装时特别要注意功放块、运算放大器、电解电容等主要器件的引脚和极性,不能接错。
	\item	从输入级开始向后级安装,也可以从功放级开始向前逐级安装。
	\item	安装一级调试一级,安装两级要进行级联调试,直到整机安装与调试完成。
\end{enumerate}

\subsubsection{电路调试技术}

\begin{enumerate}

	\item 电路的调试过程一般是先分级调试,再级联调试,最后进行整机调试与性能指标测试。
	\item 分级调试又分为静态调试与动态调试。
	      静态调试时,将输入端对地短路,用万用表测该级输出端对地的直流电压。话放、混放、音调电路均由运放组成,若运放是单电源供电,其静态输出直流电压均为$V_{CC}$/2,功放级的输出(OTL电路)也为$V_{CC}$/2,且输出电容CC两端充电电压也应为$V_{CC}$/2。若是双电源供电,直流电压均为0。
	      动态调试是指输入端接入规定的信号,用示波器观测该级输出波形,并测量各项性能指标是否满足题目要求,如果相差很大,应检查电路是否接错,元器件数值是否合乎要求,否则是不会出现很大偏差的。
	\item 级联调试
	      单级电路调试时的技术指标较容易达到,但级联后级间相互影响,可能使单级的技术指标发生很大变化,甚至两级不能进行级联。产生的主要原因:一是布线不太合理,形成级间交叉耦合,应考虑重新布线;二是级联后各级电流都要流经电源内阻,内阻压降对某一级可能形成正反馈,应接RC去耦滤波电路。R一般取几十欧姆,C一般用几百微法大电容与0.1F小电容相并联。由于功放输出信号较大,易对前级产生影响,引起自激。集成块内部电路多极点引起的正反馈易产生高频自激,常见高频自激现象如\reffig{高频自激}所示。
	      \vspace{1em}
	      \begin{figure}[H]
		      \centering
		      \includegraphics[width=1\textwidth]{高频自激.png}
		      \caption{高频自激}
		      \label{高频自激}
	      \end{figure}
\end{enumerate}

\section{实验过程}
\subsection{实际电路与功率、增益、效率}
\begin{figure}[H]
	\centering
	\includegraphics[width=1\textwidth]{1KHz工作点.PNG}
	\caption{1KHz工作点}
	\label{1KHz工作点}
\end{figure}
\noindent 增益:
\begin{equation}
	A_v = \frac{v_o}{v_i} = \frac{4.850V}{15.77mV} = 307.546
\end{equation}
输出功率:
\begin{equation}
	P_o = \frac{U^2}{R} = \frac{(4.85/2)^2}{2\times 10} = 0.294W
\end{equation}
电源功率:
\begin{equation}
	P = UI = 8.994V \times 0.081A = 0.729W
\end{equation}
电源效率:
\begin{equation}
	\eta = \frac{P_o}{P} = 40.384\%
\end{equation}
\newpage
\subsection{输入阻抗分析}
在输入端串联500kΩ电位器,输出无明显变化,可以认为 $R_i \gg 500K\Omega$,进行仿真分析得到输入阻抗和频率的关系如\reffig{输入阻抗分析}所示:
\begin{figure}[H]
	\centering
	\includegraphics[width=1\textwidth]{输入阻抗分析}
	\caption{输入阻抗分析}
	\label{输入阻抗分析}
\end{figure}
由上图及实验结果可知,输入阻抗达到 $\mathrm{M}\Omega$级别,满足设计要求。
\subsection{幅频响应}
\begin{figure}[H]
	\centering
	\includegraphics[width=1\textwidth]{下限频率}
	\caption{下限频率}
	\label{下限频率}
\end{figure}
\begin{figure}[H]
	\centering
	\includegraphics[width=1\textwidth]{20KHz工作点}
	\caption{20KHz工作点}
	\label{20KHz工作点}
\end{figure}
\begin{table}[H]
	\centering
	\large
	\begin{tabular}{|c|c|c|c|c|}
		\hline
		\textbf{频率}           & \textbf{1KHz} & \textbf{50Hz} & \textbf{$f_L =42Hz$} & \textbf{$f_H=69KHz$} \\ \hline
		$\mathrm{V_{Ipp}/mV}$ & 15.61         & 15.23         & 15.24                 & 14.73                \\ \hline
		$\mathrm{V_{Opp}/V}$  & 4.880         & 3.759         & 3.409                 & 3.406                \\ \hline
		$\mathrm{A_v}$        & 312.62        & 246.81        & 223.69                & 231.23               \\ \hline
	\end{tabular}
\end{table}
可知,下限频率 $\mathrm{f_l = 42Hz<50Hz}$,上限频率 $\mathrm{f_h=69KHz>20KHz}$,满足设计要求。


\section{实验总结}
通过本实验,我通过自己的努力完成了一个比较复杂的音响放大电路,增强了对运算放大器的理解,对集成功率放大器内部电路工作原理和应用有了更强的把握。

电路搭好的时候,发现功放级不接负载波形正常,接了负载之后电流增大,电路开始不稳定,出现自激振荡。后来在功放的输入端接了一个到地的滤波电容,输出正常,但是上限频率达不到要求。最后将滤波电容改小后,上限频率增大,达到要求。本次实验真正的难点在调试上,通过本次实验,我学到了不少调试电路的技巧。

\end{document}
